X-\/\+Tetris is an advanced version of the original Tetris game

X-\/\+Tetris è una versione avanzata del gioco originale del Tetris. Per le regole del tetris puoi visitare la pagine Wikipedia \href{https://it.wikipedia.org/wiki/Tetris}{\texttt{ https\+://it.\+wikipedia.\+org/wiki/\+Tetris}}.

Autore Alessandro Cecchin Progetto per il corso IaP dell’\+Università di Venezia Features

Single Player

A differenza del Tetris originale, in X-\/\+Tetris il giocatore ha inizialmente a disposizione 20 pezzi per ciascun tipo, detti tetramino, e una mossa consiste nello scegliere quale pezzo giocare,dove farlo cadere e con quale rotazione. Il campo di gioco è largo 10 e alto 15 posizioni. Una volta posizionato un tetramino, se una o più righe orizzontali vengono riempite per intero, queste righe vengono cancellate come nel tetris originale. La rimozione di una riga vale 1 punto, la rimozione di due righe con un solo pezzo vale 3 punti, tre righe 6 punti, quattro righe 12 punti. Il gioco termina quando finiscono i pezzi o il giocatore non riesce a posizionare un tetramino nel campo di gioco rispettando il limite di altezza e larghezza. oppure supera i 100 punti.

Multi Player

Il programma supporta una seconda modalità di gioco, selezionabile dal menu iniziale, in cui due giocatori giocano a turni alterni ciascuno nel proprio campo di gioco ma pescando dallo stesso insieme di tetramini. In questa modalità si avranno il doppio di pezzi a disposizione. Nel caso in cui un giocatore cancelli una o due linee simultaneamente, il gioco procede come per il caso single player. Nel caso il giocatore cancelli 3 o più linee con una singola mossa, il campo dell’avversario viene modificato invertendo il corrispondente numero di linee nella parte più bassa del campo di gioco\+: una posizione vuota diventa piena e viceversa. Un giocatore perde la partita se non posiziona correttamente un pezzo nel proprio campo di gioco. Se i pezzi finiscono vince il giocatore con il punteggio più alto. La modalità multi-\/player deve prevede la possibilità di giocare player vs. player

la costruzione del campo di gioco dei tetramini e di ogni cosa ha solamente utilizzato la funzione printf e le librerie standard di C

Logica di costruzione campo

Un puntatore punta ad una struttura composta da RIGHE x COLONNE riquadri. ad ogni turno il puntatore viene usato per stampare l’intero campo di gioco e per riempirlo in base alla colonna e al tetramino scelto. 